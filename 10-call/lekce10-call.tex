\documentclass{beamer}
%[aspectratio=169]   \usepackage[czech]{babel}
\usepackage{apo-lecture}
\usepackage{pdfpages}
\usepackage{pdfcomment}
\usepackage{listings}
\usepackage{array,multirow}

\subtitle{Lekce 10. Volání funkce}
\author{Pavel Píša \phantom{xxxxxxxxx} Petr Štěpán \\ \small\texttt{pisa@fel.cvut.cz}\phantom{xxxx}\small\texttt{stepan@fel.cvut.cz}}
\begin{document}

\maketitle

\section{Volání funkce}

\begin{frame}
\frametitle{Cíl dnešní přednášky}

\begin{itemize}
 \item Volání funkce
\end{itemize}
\end{frame}


\begin{frame}
\frametitle{Jak se překládá program v C}

Náhled na překlad programu v jazyce C do assembleru např. RISC-V
\begin{itemize}
 \item Překlad a = b + c;
\end{itemize}
\end{frame}


\begin{frame}
\frametitle{Překlad konstrukce while}

Porovnání překladu while konstrukce:
\begin{itemize}
 \item dva překlady while skok, 
\end{itemize}
\end{frame}


\begin{frame}
\frametitle{Jak se přeložit funkci?}

Jak předat parametry a,b?
Jak předat výsledek secti?
Co by mělo být na konci funkce, kam skočit?
\begin{itemize}
 \item secti ( int a, int b)
\end{itemize}
\end{frame}


\begin{frame}
\frametitle{Jak přeložit funkci?}

Volající (caller) a volaný (callee) se musí dohodnout na způsobu předání parametrů.
\begin{itemize}
 \item Překlad volaného může být na jiném počítači, jiným překladačem (typicky knihovny) než překlad volajícího - Vaším překladačem na Vašem počítači.
 \item Je nutné definovat konvenci, typ konvence volání funkce musí být v hlavičce objektových souborů (a tím i knohoven) aby zajistila správnost předání dat a získání návratové hodnoty.
\end{itemize}
\end{frame}



\begin{frame}
\frametitle{Jak přeložit funkci?}

Uložení parametrů do registrů
\begin{itemize}
 \item Překlad volaného může být na jiném počítači, jiným překladačem (typicky knihovny) než překlad volajícího - Vaším překladačem na Vašem počítači.
 \item Je nutné definovat konvenci, typ konvence volání funkce musí být v hlavičce objektových souborů (a tím i knohoven) aby zajistila správnost předání dat a získání návratové hodnoty.
\end{itemize}
\end{frame}


\begin{frame}
\frametitle{Jak přeložit funkci?}

Uložení návratové hodnoty do registru ra
\begin{itemize}
 \item 
\end{itemize}
\end{frame}

\begin{frame}
\frametitle{Příklad volání funkce z dvou různých adres}

Uložení návratové hodnoty do registru ra
\begin{itemize}
 \item 
\end{itemize}
\end{frame}

\begin{frame}
\frametitle{Co když funkce ve svém těle volá jinou funkci?}

Definice problému vícenásobného použití registrů
\begin{itemize}
 \item Activation record nebo také activation frame 
 \item call stack nebo stack frame
\end{itemize}
\end{frame}


\begin{frame}
\frametitle{Zásobník}

Operace push, pop:
\begin{itemize}
 \item push 
 \item pop
\end{itemize}
\end{frame}

\begin{frame}
\frametitle{Konvence volání funkce}

\begin{itemize}
 \item Předávání parametrů 
 \item Zachování hodnot vybraných registrů
 \item Lokální proměnné funkce
 \item Návratová hodnota funkce
\end{itemize}
\end{frame}


\begin{frame}
\frametitle{Konvence volání funkce}

\begin{itemize}
 \item Co zachová volající -- caller 
 \item Co zachová volaný -- callee
\end{itemize}
\end{frame}

\begin{frame}
\frametitle{Konvence volání funkce}

\begin{itemize}
 \item Předávání parametrů, co když se parametry nevejdou do registrů, pole
 \item Co když se výsledek nevejde do dvou registrů 
\end{itemize}
\end{frame}


\end{document}

