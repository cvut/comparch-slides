\documentclass{beamer}
%[aspectratio=169]   \usepackage[czech]{babel}
\usepackage{apo-lecture}
\usepackage{pdfpages}
\usepackage{pdfcomment}
\usepackage{listings}
\usepackage{array,multirow}

\subtitle{Lekce 01. Úvod}
\author{Petr Štěpán\\ \small\texttt{stepan@fel.cvut.cz}}
\begin{document}

\maketitle

\section{Obsah}

\begin{frame}
\frametitle{Motivace}
Proč studovat architektury počítačů:
\begin{itemize}
\item Poptávka po absolventech kombinující umělou inteligenci a vestavné systémy (embedded system)
\item Tvorba efektivních programů využívající výhody výpočetní techniky
\end{itemize}

\end{frame}


\begin{frame}
\frametitle{Obsah přednášek}
\begin{itemize}
\item CPU
\item Hierarchie paměti - Cache/RAM
\item Vstupy a výstupy - I/O
\item Výjimky a přerušení
\end{itemize}
\end{frame}

\begin{frame}
\frametitle{Náplň cvičení}

\begin{itemize}
\item 4 menší domácí úkoly - 36 bodů
\begin{itemize}
\item 2 programy v C
\item 2 formuláře
\item alespoň 3 úlohy ze 4
\end{itemize}
\item Semestrální úloha - 24 bodů
\begin{itemize}
\item Týmový projekt - dvojice, nebo jednotlivci
\item Speciální HW MZ APO deska
\end{itemize}
\item Nepoviné úlohy nebo aktivita při cvičení - 8 bodů
\end{itemize}

Zkouška: test 30 bodů, min 15 bodů
ústní +-10 bodů    
\end{frame}


\begin{frame}
\frametitle{Navazující předměty}

\begin{itemize}
\item B4M35PAP - Pokročilé architektury počítačů
\item B3B38VSY - Vestavné systémy
\item B4M38AVS - Aplikace vestavných systémů
\item B4B35OSY - Operační systémy
\item B0B35LSP - Logické systémy a procesory
\end{itemize}
\end{frame}


\begin{frame}
\frametitle{Materiály k předmětu}
\begin{itemize}
\item Paterson, D., Hennessey, V.: Computer Organization and Design, The HW/SW Interface. Elsevier, ISBN: 978-0-12-370606-5 (dostupná v knihovně FEL)
\item web:
\begin{itemize}
\item https://cw.fel.cvut.cz/b192/courses/b35apo/
\item https://dcenet.felk.cvut.cz/apo/
\end{itemize}
\item Kurzy v angličtině:
\begin{itemize}
\item MIT 6.004/6.191 – Computation Structures
\item Computation Structures | Electrical Engineering and Computer Science | MIT OpenCourseWare (2015)
\item Computer System Architecture | Electrical Engineering and Computer Science | MIT OpenCourseWare (2005)
\end{itemize}
\item Kurzy v češtině:
\begin{itemize}
\item https://courses.fit.cvut.cz/BI-APS/
\item https://www.vut.cz/studenti/predmety/detail/218515?apid=218515
\end{itemize}

\end{itemize}
\end{frame}


\begin{frame}
\frametitle{Co je uvnitř počítače}

Základní deska počítače
\end{frame}

\begin{frame}
\frametitle{Co je uvnitř počítače}

Rozebraný telefon
\end{frame}


\begin{frame}
\frametitle{von Neumann}

CPU, Paměť, Vstup/Výstup

\end{frame}

\begin{frame}
\frametitle{von Neumann}

Společný koncept

Obrázek

\end{frame}


\begin{frame}
\frametitle{CPU}

CPU provádí instrukce
\end{frame}

\begin{frame}
\frametitle{Paměť}

Pamatuje si data - bajty, slova.
\end{frame}

\begin{frame}
\frametitle{Boolova algebra}

False True 0/1 nesvítí/svítí - úroveň napětí
\end{frame}

\begin{frame}
\frametitle{Boolova algebra}

Základní operace not, and, or
\end{frame}

\begin{frame}
\frametitle{Boolova algebra}

Obvod not, and, or
\end{frame}

\begin{frame}
\frametitle{Boolova algebra}

Rozšířené operace nand, nor, xor
\end{frame}

\begin{frame}
\frametitle{Boolova algebra}

Složitější obvody - složené ze základních operací, stavebních bloků

Příklad
\end{frame}

\begin{frame}
\frametitle{Boolova algebra}

xor a and je vlastně součet
\end{frame}

\begin{frame}
\frametitle{Binární soustava}

Více bitová čísla - binární soustava
\end{frame}

\begin{frame}
\frametitle{Sčítání}

Half/Full adder
\end{frame}

\begin{frame}
\frametitle{Sčítání}

Ripple carry adder
\end{frame}

\begin{frame}
\frametitle{Sčítání}

Rychlost - Ripple carry adder
\end{frame}

\begin{frame}
\frametitle{Sčítání}

Look ahead adder
\end{frame}

\begin{frame}
\frametitle{Sčítání}

Look ahead adder
\end{frame}

\begin{frame}
\frametitle{Sčítání}

Look ahead adder
\end{frame}

\begin{frame}
\frametitle{Sčítání}

Look ahead adder
\end{frame}

\begin{frame}
\frametitle{Sčítání}

Look ahead adder
\end{frame}

\end{document}

