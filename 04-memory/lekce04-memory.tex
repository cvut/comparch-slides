\documentclass{beamer}
%[aspectratio=169]   \usepackage[czech]{babel}
\usepackage{apo-lecture}
\usepackage{pdfpages}
\usepackage{pdfcomment}
\usepackage{listings}
\usepackage{array,multirow}

\subtitle{Lekce 04. Hierarchie paměti}
\author{Pavel Píša \phantom{xxxxxxx} Petr Štěpán \\ \small\texttt{pisa@fel.cvut.cz}\phantom{xxxx}\small\texttt{stepan@fel.cvut.cz}}
\begin{document}

\maketitle

\section{Druhy paměti}

\begin{frame}[fragile]
\frametitle{Motivace}

\begin{columns}
\begin{column}{0.45\textwidth}
Algoritmus A\\
\begin{minted}{c}
int matrix[N][N];
int main() {
  long int i, j, sum1 = 0;
  for(i=0; i<N; i++)
    for(j=0; j<N; j++)
      sum1 += matrix[i][j];
}
\end{minted}
\end{column}
\hfill
\begin{column}{0.45\textwidth}
Algoritmus B\\
\begin{minted}{c}
int matrix[N][N];
int main() {
  long int i, j, sum1 = 0;
  for(i=0; i<N; i++)
    for(j=0; j<N; j++)
      sum1 += matrix[j][i];
}
\end{minted}
\end{column}
\end{columns}

\begin{tabular}{|l|l|l|}\hline
N & A & B \\\hline
100000 & 12.791328s &138.047563s \\\hline
10000 & 0.126945s &0.486535s \\\hline
1000 & 0.001329s &0.001756s \\\hline
100 & 0.000083s &0.000094s \\\hline
\end{tabular}

\end{frame}


\begin{frame}
\frametitle{Druhy paměti}

\begin{itemize}
\item Různé druhy paměti
\end{itemize}

\end{frame}


\end{document}

