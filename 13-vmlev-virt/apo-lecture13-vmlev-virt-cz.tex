\documentclass{beamer}
%[aspectratio=169]   \usepackage[czech]{babel}
\usepackage{apo-lecture-cz}
\usepackage{pdfpages}
\usepackage{pdfcomment}
\usepackage{listings}
\usepackage{array,multirow}
\usepackage{environ}

\subtitle{Lekce 13. Virtuální stroje a virtualizace}
\author{Pavel Píša \phantom{xxxxxxxxx} Petr Štěpán \\ \small\texttt{pisa@fel.cvut.cz}\phantom{xxxx}\small\texttt{stepan@fel.cvut.cz}}

\begin{document}

\maketitle

\section{Mnohaúrovňová organizace počítače}

\begin{frame}
\frametitle{Cíl dnešní přednášky}

\begin{itemize}
 \item Mnohaúrovňová organizace počítače
 \item Virtualizace
\end{itemize}
\end{frame}

\begin{frame}
\frametitle{Mnohaúrovňová organizace počítače}

\begin{itemize}
 \item Strojový jazyk počítače
 \begin{itemize}
  \item množina jednotlivých instrukcí
  \item úroveň L1 -- abeceda {0,1}
  \item do ní je potřeba převést program pro vykonání
  \item pro návrh aplikace člověkem i generátorem vhodnější jazyky vyšší úrovně L2 + další
 \end{itemize}
 \item Vykonání programu v L2 na stroji jenž poskytuje L1
 \begin{itemize}
  \item Kompilace -- instrukce v L2 se nahradí posloupností instrukcí v L1
  \item Interpretace -- program v L1 zpracovává program v L2 jako data (pomalejší)
 \end{itemize}
 \item Vývoj víceúrovňového stroje
 \begin{itemize}
  \item první počítače -- běžná strojová úroveň - 1.úr.
  \item 50-tá léta -- Wilkes -- mikroprogram. - 2.úr.
  \item 60-tá léta -- operační systémy - 3.úr.
  \item překladače, program. jazyky - 4.úr.
  \item uživatelské aplikační programy - 5.úr.
  \item problémově orientované jazyky a prostředí
 \end{itemize}
\end{itemize}

\end{frame}

\begin{frame}
\frametitle{Virtuální počítač}

Technické prostředky (HW) a programoví vybavení (SW) jsou logicky ekvivalentní (lze je navzájem nahradit) souboj a splývání  RISC a CISC procesorů
\vspace{0.2cm}
Na úrovni $i$ zavádíme virtuální počítač $M_i$ s jazykem $L_i$.
Program v $L_i$ je překládán nebo interpretován počítačem $M_{i-1}$ atd.

\begin{center}
\includegraphics[width=0.75\textwidth]{mutilevel-machine-cz.pdf}
\end{center}

\end{frame}

\begin{frame}
\frametitle{Současná mnohaúrovňová architektura počítače}

\begin{center}
\includegraphics[width=0.75\textwidth]{mutilevel-mcode-to-highlev-cz.pdf}
\end{center}

\end{frame}

\begin{frame}
\frametitle{Procesy a jejich stavy}

\begin{description}
 \item[proces] probíhající program (program -- pasivní, proces -- aktivní)
 \item[stav procesu] informace, které při zastavení procesu umožní jeho znovuspuštění
 \begin{enumerate}
  \item program
  \item násled. instrukce
  \item hodnoty proměnných a data
  \item stavy a polohy všech V/V (pozice v otevřených souborech atd..)
  \item rozložení/obsazení adresního prostoru (paměťový kontext)
 \end{enumerate}
 \item[předpoklad] proces P sám nemění svůj program, pokud ano, tak jsou změněné části součástí stavu, dat
 \item[stavový vektor] proměnné složky stavu procesu -- mění je program, OS nebo HW
   $$PROCES = PROGRAM + STAVOVÝ \medskip  VEKTOR$$
\end{description}
\end{frame}

\begin{frame}
\frametitle{Konvenční strojová úroveň (ISA) (M2)}

Definovaná Instrukční sadou (často označovaná architektura procesoru) ISA -- Instruction Set Architecture $\times$ Mikroarchitektura (Implementace v M1 nebo přímo v logickém návrhu procesoru)

\begin{itemize}
 \item struktura počítače, procesoru, V/V kanálů, sběrnic, organice a přístupy k paměti, organizace registrů a další
 \item instrukční soubor, formát dat a uložení v paměti, adresování, zápisníková paměť (registry)
\end{itemize}

Instrukční formát

\begin{itemize}
 \item skládá se z polí uložených v jednom nebo více po sobě jdoucí slovech
 \item operační znak (opcode)
 \item přímé operandy (immediate), konstanty vložené do instrukce
 \item adresové části mohou být adresami do paměti nebo odkazy, čísla registrů a ty použity buď přímo nebo jako adresy do paměti (nepřímá adresace)
\end{itemize}

\end{frame}

\section{Virtualizace}

\begin{frame}
\frametitle{Virtualizace}

\begin{itemize}
 \item Virtualizace skrývá implementaci/vlastnosti nižších vrstev (reality) a předkládá prostředí s požadovanými vlastnostmi
 \item V počítačové technice rozdělujeme virtualizaci na
 \begin{itemize}
  \item Čistě aplikační/na úrovni jazyků a kompilovaného kódu (byte-kód) – virtuální stroje, např. JVM, dotNET
  \item Emulace a simulace typicky jiné počítačové architektury (také zvaná křížová virtualizace)
  \item Nativní virtualizace – izolované prostředí poskytující shodný typ architektury pro nemodifikovaný OS
  \item Virtualizace s plnou podporou přímo v HW
  \item Částečná virtualizace -- typicky jen adresní prostory
  \item Paravirtualizace -- systém musí být pro běh v nabízeném prostředí upraven
  \item Virtualizace na úrovni OS -- pouze oddělená uživ. prostředí (kontejnery)
 \end{itemize}
\end{itemize}
\end{frame}

\begin{frame}
\frametitle{Virtualizace na úrovni operačního systému a aplikací}

\begin{itemize}
 \item I vlákna lze brát jako virtualizaci stavu procesoru a procesy jako virtualizaci celého prostředí včetně oddělení paměti
 \item Virtuální stroje a interpretery v rámci aplikací (prohlížeč, Python v LibreOffice, atd.)
 \begin{itemize}
  \item přímá interpretace v textové podobě málo častá, většinou překlad do binární reprezentace (bytecode) a z té často i do nativního kódu (JVM)
 \end{itemize}
 \item Kontejnery
 \begin{itemize}
  \item chroot -- samostatný pohled na adresářovou strukturu (Mount -- mnt)
  \item LXC (Linux Containers) -- oddělitelnost různých jmenných prostorů jádra (namespace) a prioritizace prostředků (cgroups)
  \begin{itemize}
   \item Mount (mnt), Process ID (pid), Network (net), Inter-process Communication (ipc), UTS (host+domain), User ID (user), Control group (cgroup) Namespace, Time Namespace
  \end{itemize}
  \item Docker -- většinou nad LXC
  \item systemd-nspawn
  \item Solaris Containers (Zones)
  \item FreeBSD jail
 \end{itemize}
\end{itemize}
\end{frame}

\begin{frame}
\frametitle{Virtualizace na úrovni celého stroje}

\begin{itemize}
 \item Hostitelský počítač (Host, Domain DOM 0)
 \item Hostovaný systém (Guest)
 \item Procesor pro hostovaný systém
 \begin{itemize}
  \item pro nativní případ běžný kód/neprivilegované instrukce zpracovány přímo CPU
  \item pro křížový případ interpretace instrukcí emulátorem (program v DOM 0), případně akcelerace
 \end{itemize}
 \item Privilegované instrukce v hostovaném systému
 \begin{itemize}
  \item způsobí výjimky, které obslouží monitor/hypervizor tak, že je odemuluje
  \item CPU má podporu pro HW virtualizaci (AMD-V, Intel VT-x), stínové stránkovací tabulky
 \end{itemize}
 \item Periferie/zařízení
 \begin{itemize}
  \item přístupu na IO a paměťově mapované periferie způsobují výjimky a hypervizor odsimuluje jejich činnost
  \item hostovaný systém se přizpůsobí tak, aby předal požadavky přímo ve formátu, kterému hypervizor rozumí (ovladače na míru atd.)
 \end{itemize}
\end{itemize}
\end{frame}

\begin{frame}
\frametitle{Hypervizor}

\begin{itemize}
 \item zajišťuje spouštění a zastavování domén
 \item monitoruje jejich činnost a ošetřuje výjimky – především emuluje činnost privilegovaných instrukcí
 \item zajišťuje rozdělení paměti a výpočetního výkonu do hostovaných systémů
 \item emuluje činnost periferií a předává data do ovladačů fyzických zařízení a sítí na úrovni hostitelského systému
 \item může být implementovaný
 \begin{itemize}
  \item v uživatelském prostoru hostitelského systému (QEMU)
  \item s využitím podpory v HW a jádře OS (KVM)
  \item jako samostatný systém/mikrojádro, který využívá systém v jedné doméně (DOM 0) pro komunikaci s fyzickými zařízeními a z tohoto systému data přeposílá do ostatních (XEN)
 \end{itemize}
\end{itemize}
\end{frame}

\end{document}

