\documentclass{beamer}
%[aspectratio=169]   \usepackage[czech]{babel}
\usepackage{apo-lecture}
\usepackage{pdfpages}
\usepackage{pdfcomment}
\usepackage{listings}
\usepackage{array,multirow}

\subtitle{Lekce 02. Reprezentace čísel}
\author{Petr Štěpán\\ \small\texttt{stepan@fel.cvut.cz}}
\begin{document}

\maketitle

\section{Opakování}


\begin{frame}
\frametitle{Opakování}
V minulé lekci jsme měli:
\begin{itemize}
\item Reprezentace bitu pomocí napětí
\item Reprezentace bajtů jako více paralelních vodičů, každý s hodnotou jednoho bitu
\item Sčítání dvou celých kladných čísel
\item Posun celých čísel (násobení, nebo dělení mocninou 2)
\end{itemize}

Dnes:
\begin{itemize}
\item Rozsahy celých čísel a ukládání do paměti
\item Násobení a dělení celých kladných čísel 
\item Reprezentace záporných čísel a operace s nimi
\item Přetečení sčítání a odčístání
\item Reálná čísla
\end{itemize}

\end{frame}


\begin{frame}
\frametitle{Kladná čísla}
Reprezentace celých kladných čísel
\begin{itemize}
\item Rozsahy pro 1,2,4,8 bajtů
\item Dekadické zápisy, hexa zápisy
\end{itemize}

\end{frame}


\begin{frame}
\frametitle{Kladná čísla}

Ukládání čísel do paměti Big endian / Little endian
\end{frame}



\begin{frame}
\frametitle{Násobení celých čísel}

Obdoba násobení jak jste se ho nauřili na základní škole
\end{frame}

\begin{frame}
\frametitle{Násobení celých čísel}

Násobička s posuvným registrem

Pomalá
\end{frame}


\begin{frame}
\frametitle{Násobení celých čísel}

Rychlé násobení celých čísel Wallace tree

Motivace rychlé sečtení čtyř 64-bitových čísel
\end{frame}

\begin{frame}
\frametitle{Násobení celých čísel}

Rychlé násobení celých čísel Wallace tree
\end{frame}

\begin{frame}
\frametitle{Násobení celých čísel}

Rychlé násobení celých čísel Wallace tree

Shrnutí rychlosti násobení 64-bitových čísel
\end{frame}


\begin{frame}
\frametitle{Dělení celých čísel}

Obdoba násobení, nalezneme i zbytek po dělení

\end{frame}


\begin{frame}
\frametitle{Dělení celých čísel}

Dělička - existuje rychlejší algoritmus, ale je velmi složitý

\end{frame}


\section{Záporná čísla}
\begin{frame}
\frametitle{Reprezentace záporných čísel}
Doplněk dvou

\end{frame}

\begin{frame}
\frametitle{Opačné číslo}

Jak z X udělat -X

\end{frame}


\begin{frame}
\frametitle{Počítání se zápornými čísly}

Výhoda reprezentace doplněk dvou, nemusíme vymýšlet nové sčítání, funguje stejný algoritmus
\end{frame}

\begin{frame}
\frametitle{Odčítání}

Přičteme opačné číslo a je to.
\end{frame}

\begin{frame}
\frametitle{Přetečení}

Co se stane, když spustíte následujcí C program:

\end{frame}

\begin{frame}
\frametitle{Přetečení}

Přetečení při sčítání kladnách čísel.
\end{frame}

\begin{frame}
\frametitle{Přetečení}

Přetečení při sčítání Záporných čísel.
\end{frame}

\begin{frame}
\frametitle{Násobení a dělení}

Počítáme s absolutními hodntami a nakonec určíme znaménko výsledku podle znaménka operandů.
\end{frame}

\begin{frame}
\frametitle{Jiné reprezentace záporných čísel}

Čísla s posunutou nulou
\end{frame}

\begin{frame}
\frametitle{Počítání s posunutou nulou}

Čísla s posunutou nulou
\end{frame}


\begin{frame}
\frametitle{Jiné reprezentace záporných čísel}

Doplněk jedné

BCD formát
\end{frame}


\section{Reálná čísla}


\begin{frame}
\frametitle{Reálná čísla}

Dvojková reálná čísla

\end{frame}


\begin{frame}
\frametitle{Reálná čísla}

Čísla s pevnou destinnou čárkou

Obdoba čísel s posunutou nulou

Jednoduché sčítání odčístání, složité násobení, dělení
\end{frame}

\begin{frame}
\frametitle{Plovoucí desetinná čárka}

Plovoucí desetinná čárka, floating point numbers

Obdoba dekadického zápisu reálných čísel

\end{frame}

\begin{frame}
\frametitle{IEEE-754}

Základní definice znaménko exponent mantisa (skrytá jednička) 
\end{frame}


\begin{frame}
\frametitle{IEEE-754}

Normalizované, denormalizované čísla

Nekonečna a NaN
\end{frame}

\begin{frame}
\frametitle{IEEE-754}

Porovnání dvou reálných čísel

\end{frame}


\begin{frame}
\frametitle{IEEE-754}

Sčítání dvou reálných čísel

\end{frame}


\begin{frame}
\frametitle{IEEE-754}

Sčítání dvou reálných čísel

\end{frame}

\begin{frame}
\frametitle{IEEE-754}

Násobení dvou reálných čísel

\end{frame}

\begin{frame}
\frametitle{Bonusový bod}

Kvíz s bonusovou otázkou za tuto hodinu.

\end{frame}




\end{document}

